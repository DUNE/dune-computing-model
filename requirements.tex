\fxnote{Import the Requirements with necessary edits}
\subsection{Purpose, Origin and Scope}

These Requirements have been developed based in part on prior planning work done for the Long-Baseline Neutrino Experiment (LBNE), which is a predecessor of DUNE.
They constitute one of principal components of the Computing Model. 

This section does not cover specific \textit{functional requirements} for the various Physics Tool software but rather focuses on the foundations of the Software and Computing infrastructure, such as code management, Grid capability, data management etc (and does not concern itself specifically with the application code such as tracking, analysis etc).

An effort was made to maintain a relatively high-level view of the DUNE computing issues, and to not go into smaller details which are more likely to change as the project moves forward, while still providing adequate basis for making informed decisions. In cases where it was impossible to establish concrete metrics or parameters for a specific requirement, it is still listed as an item
to be addressed in the future.

Information necessary for the creation of these Requirements have been collected during a number of LBNE Software and Computing meetings, conference calls and extensive information exchange via e-mail and other means. It is recognized that the Requirements themselves (and the Computing Model) may evolve with time, in order to correctly reflect the status of rapidly evolving technologies and the LBNE organization itself. Such expectation is reinforced by actual experience of large scale HEP collaborations. We anticipate therefore that the requirements will be revised roughly in the middle of the time period between their creation and the commissioning of the experiment, and help evolve the LBNE Computing model such that it continues to meet the needs of the Collaboration.

Certain  requirements need to be specified by a collaboration including the S\&C Organization and other software organizations in LBNE (i.e. the Physics Tools Group, the Online/DAQ groups etc). In these cases, the S\&C Organization will work with the appropriate group to insure that the requirement addresses the needs of all necessary organizations.  This includes interfacing with various LBNE Project groups in order to assure the seamless cooperation between the online and offline areas.
