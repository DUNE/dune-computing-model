\subsection{Offline Software}
\subsubsection{Overview}

For s detailed presentation of requirements to DUNE offline software please see Section~\ref{sec:req-software}.
A few important items which will be pursued in DUNE are achitectural compliance, code review, revision control, release methodology,
 and documentation. Due to the large size of the Collaboration and diversity of its distributed
 computing resources DUNE will have to create a ``supported platform'' policy which will define the mechanisms by
which platforms are selected for support, retired from
supported status and what level of support and methods for its delivery will be provided.

Just like the vast majority of other physics experiments in HEP and Intensity Frontier, DUNE relies on the ROOT\cite{root}
system as the foundation of a wide spectrum of its software. It forms the basis for the \textit{art} framework\cite{art},
\textit{artdaq} -- a DAQ framework which is an extension of \textit{art},
and LArSoft -- a software suite for simulation and reconstruction of
events in Liquid Argon TPC, which also also built on top of \textit{art}.

GEANT 4~\cite{geant4} is used in a variety of application in DUNE as the main simulation engine, and it integrated
into LArSoft and other simulation tools. MARS is used for estimation of complex radiological backgrounds and for
beam line simulation.

\subsubsection{Release managememt and Continuous Integration}
At the time of writing DUNE is already investing effort in managing releases of the LArSoft software.
Continuous Integration (CI) services are provided
by FNAL's Scientific Computing Division, and in particular Jenkins \cite{jenkins} system is utilized for CI.

One of the challenges in managing DUNE software is active use of LArSoft across a few experiments in the
Intensity Frontier, which constantly leads to new requirements, features and otherwise results in updates
of the design and interfaces of this software suite. Different branches may have compatibility issues.

\subsubsection{Software provisioning}
DUNE (like its predecessor LBNE) is using a few different methods to deploy its software on sites outside of FNAL,
both for interactive use by delvelopers and researchers and to provision software to  Worker Nodes
in batch system configuration. Currently products like \textit{art} and LArSoft can be built from source (including
the components they depend on, e.g. ROOT) using utilities created and maintained by the FNAL Scientific Computing
Division. This method can be used to make this software available on a cluster or interactive workstations outside
of FNAL.

DUNE is following the common trend of delivering software via the networks, in particular experience has been gained with
CernVM-FS (Cern-VM File System, commonly called CVMFS) \cite{cvmfs}. It is a network file system based on HTTP and
optimized to deliver experiment software in a fast, scalable,
and reliable way. It was originally designed to decouple the life cycle management of the application software releases
from the operating system of a specific configuration of a Virtual Machine, ``CERNVM''. Now it is often used independently
from its original purpose and has proven itself as efficient tool for software delivery to worker nodes in systems and locations
where complete installation (``build'') of DUNE software is not practical or desirable.