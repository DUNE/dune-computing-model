\subsection{DUNE}

The Deep Underground Neutrino Experiment (DUNE) will provide a unique, world-leading program
for the exploration of key questions at the forefront of particle physics and astrophysics (ref. to sciopps
and CD1R). DUNE will involve a number of subsystems and components. From the computing
perspective, two can be qualified as especially important:

\begin{enumerate}

\item a fine-grained \textbf{Near Neutrino Detector} just downstream of the neutrino source
\item a massive liquid argon time-projection chamber (LArTPC) deployed as a \textbf{Far
Detector} deep underground, 1,300 km downstream from the source of neutrinos located at FNAL

\end{enumerate}

The Near Detector will comprise a ``fine-gran tracker'' (FGT) and a calorimeter, with channel count of the order of $10^{5}$.

The Far Detector will be installed at the Sanford Underground Research Facility in Lead, South Dakota.
Its sensitive volume will be formed by a total of 40kt of Liquid Argon instrumented with $\sim$1.5 million readout
channels to detect ionization. It will also incorporate a photon detector which will detect scintillation
signal produced by ionizing patciles traveling in Liquid Argon.



\subsection{Computing Model: From Requrements to Implementation}
\label{sec:modelrole}

The Computing Model has been developed to address the \textit{Software and Computing Requirements}, which were created over a period 
of a few months in early 2014 and reflect a broad consensus within the Collaboration regarding what it needs from its computing sector and its components. In turn, it serves as the basis for the Implementation Plan, which will be used to create concrete work breakdown schedules and resource allocation plans.

The LBNE Computing Model plays central role in the design of the computing infrastructure of the experiment, planning and organization of its 
Software and Computing effort, as well as formulation and implementation of policies applied in every work area. It aims to guide 
the evolution of the LBNE computing platform in the direction optimal for achieving its scientific goals. It will also serve
as an instrument to inform the funding agencies about the scope of work and rate of progress of the LBNE Software and 
Computing Organization.

\subsection{Anticipated Evolution of the Computing Model and Implementation Plan}

At the time of writing, LBNE is in the stage of broad and active R\&D effort, supported by the development of appropriate 
software tools in each area. Extensive simulations are being performed to support design decisions pertaining to the Liquid 
Argon TPC as well as a number of other detector elements and subsystems (cf. the proposed Near Detector, the Photon Detector, 
the target complex etc). Background characterization, track and event reconstruction techniques are all under active development. 
The landscape of LBNE computing will be defined by these and similar tasks for  the next few years, after which the focus will 
gradually shift towards preparations for actual data taking and analysis. 
This will include scaling up the data management and processing systems, integration testing, 
stress testing and  realistic data challenges preceding the commencement of operations.

In order to maintain a well defined structure of our planning documents, it was decided to address the issue of evolution of the Computing Model 
and the plans based on it by using the following approximation:

\begin{itemize}

\item The time period between the year 2014 and the commissioning of LBNE (its timing being uncertain as of now but assumed to be around 2026) is divided into two parts of roughly equal duration, i.e. 2014-2020 and 2020-2026. This segments will be referred to as \textbf{T1} and \textbf{T2}.


\item Major sections of the Computing Model and Implementation Plan will contain separate subsections reflecting these two time periods.

\end{itemize}
 
It is obvious that rapid evolution of software and computing technologies, combined with dynamic nature of R\&D in LBNE (and other  projects with potential impact on LBNE) and a variety of other factors makes precise formulation of the Computing Model and subsequent plans very difficult over the foreseen extended timeline. It  therefore likely that it will become necessary to evaluate the Computing Model and make adjustments at some point in time. We estimate that the transition between T1 and T2 periods (i.e. roughly 2020) will be an appropriate point in time to revisit the Computing Mode, develop its next
iteration and adjust the plans derived from it.

