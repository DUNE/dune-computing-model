\section{Introduction}
\subsection{DUNE}

The Deep Underground Neutrino Experiment (DUNE) will provide a unique, world-leading program
for the exploration of key questions at the forefront of particle physics and astrophysics~\cite{sciopps}.
DUNE will involve a number of subsystems and components. From the computing perspective, two can be
qualified as especially important:

\begin{enumerate}

\item a fine-grained \textbf{Near Neutrino Detector} just downstream of the neutrino source
\item a massive liquid argon time-projection chamber (LArTPC) deployed as a \textbf{Far
Detector} deep underground, 1,300 km downstream from the source of neutrinos located at FNAL

\end{enumerate}

The Near Detector will comprise a ``fine-gran tracker'' (FGT) and a calorimeter, with channel count of the order of O($10^{5}$).
The Far Detector will be installed at the Sanford Underground Research Facility in Lead, South Dakota.
Its sensitive volume will be formed by a total of 40kt of Liquid Argon instrumented with $\sim$1.5 million readout
channels to detect ionization signal resulting from interactions in the detector. It will also incorporate a photon detector which will detect scintillation
signal due to ionizing particles traveling within the Liquid Argon volume.

Providing extensive detail on the physics, design and operation of DUNE detectors  is beyond the scope of the Computing Model.
At the time of writing, the definitive source of such information is the DUNE Conceptual Design Report Vol.4 ~\cite{cdr_vol4_docdb}, and 
it was used to set direction of the present document.


\subsection{About this Document}
\label{sec:modelrole}

This document has the \textit{Computing Model} (Sec.\ref{sec:computing_model}) as its central part, and also
includes supporting materials that help quantify parameters of the model and explain its policies and
chosen technical solutions. The supporting materials contain two principal parts formatted as separate sections:
\begin{itemize}
\item Software and Computing Requirements, Section~\ref{sec:requirements}
\item DUNE Data Characteristics, Section~\ref{sec:data-characteristics}
\end{itemize}

\noindent
The former defines a set of rules and principles by which technologies are selected and specific practices and policies implemented in DUNE computing.
The benefit of having the \textit{Requirements} in place is that they reflect a broad consensus within the Collaboration regarding what it needs from
its computing sector and its components, thus fostering cooperation and better integrations of systems and components in DUNE.

The latter part describes the characteristics of the data to be collected, processed and otherwise generated in the experiment.
This kind of information is the foundation of any computing model.
For example, raw data rates generated by DUNE detector systems will dictate the scale and required capabilities of the Data
Acquisition System (DAQ).  Detailed discussion of the details of DAQ design and performance is far outside the scope of this
document, but essential information for the Far Detector and Near Detector Data Acquisition is presented in Sections~\ref{sec:daq}
and~\ref{sec:daq-nd} respectively.

Characteristics of the data coming out of DAQ will determine the scale of data storage, network
and other characteristics of the distributed computing system required by DUNE. Another factor in this will be the scope and projected
volume of Monte Carlo studies  will affect the scale of CPU and storage resources needed by DUNE.  Scope and complexity of the DUNE
reconstruction software will have an effect on both experimental data processing and Monte Carlo studies, and therefore of the
Computing Model as well.

This structure of the document allows a considerable amount of detail to be included in an organized fashion,
while making the Computing Model iteself a more concise and readable source of information, with
references to details provided as necessary, pointing at the supporting materials.

The Computing Model lays out the general design of DUNE computing infrastructure as well as planning and
organization of its software development effort. It aims to guide the evolution of the DUNE computing
platform in the direction optimal for achieving the scientific goals of the experiment, and will also serve
to inform funding agencies about the scope of work, resource requirements and rate of progress of 
DUNE Software and Computing sector.

There is a glossary of terms and abbreviations provided in Appendix~\ref{sec:appendix-glossary}. In addition,
there are ``Definitions'' subsections in the Requirements section of the document~(Sec.\ref{sec:requirements}) where
many relevant terms are defined.

\subsection{Timeline and Evolution of the Computing Model}
According to current plans, DUNE will be commissioned in mid-2020s. The decade before the commissioning will be quite busy
for DUNE, which is currently in the stage of broad and active R\&D effort, supported by the development of appropriate 
software tools in each area. Extensive simulations are being performed to support design decisions pertaining to the Liquid 
Argon TPC as well as a number of other detector elements and subsystems (cf. the proposed Near Detector, the Photon Detector, 
the target complex etc). Background characterization, event reconstruction techniques and studies are systematic errors are all under active development.

At the time of writing a very high priority in DUNE is assigned to two LArTPC prototypes:
\begin{itemize}
\item The ``35t prototype'' --- a Liquid Argon TPC situated at FNAL which will start taking cosmic ray data in 2016
\item The ``Full Scale'' prototype (``protoDUNE/NP04'') to be placed in a test beam at CERN, scheduled for commissioning in 2017 and actual data taking in early 2018
\end{itemize}
\noindent
It is obvious that although commissioning and operation of the eventual complete detector is still years away, DUNE is effectviely
already engaged in a substantial experimental program which poses non-trivial requirements from the computing standpoint. The
landscape of DUNE computing will be defined by these and similar tasks for the next few years.

After the crucial prototyping stage is completed some time before the year 2020, DUNE will start preparations for data taking with the
actual full apparatus, including the Far Detector at the Sanford facility and the Near Neutrino Detector and its associated systems
located at FNAL. It will involce scaling up the data management and processing systems,
integration testing,  stress testing and  realistic data challenges which must precedethe commencement of operations.
It is obvious therefore that the DUNE Computing Model will undergo considerable evolution in the following decade. In the present document,
this is addressed  by having separate sections for the computing for DUNE prototypes, and for the final DUNE experiment itself.

Aside from this obvious distinction, this Computing Model is essentially a living document: 
the rapid evolution of modern software and computing technologies, combined with dynamic nature of R\&D in DUNE
and a variety of other factors make it likely that it will become necessary to re-evaluate the Computing Model and make adjustments at
some point in time. We estimate that this may happen roughly around 2020.

