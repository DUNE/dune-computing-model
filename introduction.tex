\subsection{DUNE}

The Deep Underground Neutrino Experiment (DUNE) will provide a unique, world-leading program
for the exploration of key questions at the forefront of particle physics and astrophysics~\cite{sciopps}.

DUNE will involve a number of subsystems and components. From the computing
perspective, two can be qualified as especially important:

\begin{enumerate}

\item a fine-grained \textbf{Near Neutrino Detector} just downstream of the neutrino source
\item a massive liquid argon time-projection chamber (LArTPC) deployed as a \textbf{Far
Detector} deep underground, 1,300 km downstream from the source of neutrinos located at FNAL

\end{enumerate}

The Near Detector will comprise a ``fine-gran tracker'' (FGT) and a calorimeter, with channel count of the order of O($10^{5}$).

The Far Detector will be installed at the Sanford Underground Research Facility in Lead, South Dakota.
Its sensitive volume will be formed by a total of 40kt of Liquid Argon instrumented with $\sim$1.5 million readout
channels to detect ionization. It will also incorporate a photon detector which will detect scintillation
signal produced by ionizing patciles traveling in Liquid Argon.



\subsection{About the Computing Model}
\label{sec:modelrole}

The DUNE Computing Model plays central role in the design of the computing infrastructure of the experiment, as well as
planning and organization of its software effort. It aims to guide the evolution of the DUNE computing platform in the direction
optimal for achieving its scientific goals. It will also serve
as an instrument to inform the funding agencies about the scope of work and rate of progress of the DUNE  
Computing Organization.

The Computing Model contains the \textit{Software and Computing Requirements} that define a set of rules and principles by which technologies
are selected and specific practices and policies implemented in DUNE computing. The benefit of having the \textit{Requirements} in place is that they reflect a broad consensus
within the Collaboration regarding what it needs from its computing sector and its components, thus fostering cooperation and better
integrations of systems and components in DUNE.

Another factor that defines the structure and content of the Computing Model are the characteristics of the data to be collected,
processed and otherwise generated in the experiment. For example, raw data rates will dictate the scale and required capabilities
of the Data Acquisition System, which in turn will determine the scale and other characteristics of offline systems required by DUNE.
The scope and projected volume of Monte Carlo studies will set the scale of CPU and storage resources needed.



\subsection{The DUNE Timeline and Evolution of the Computing Model}
According to current plans, DUNE will be commissioned in mid-2020s. The decade before the commissioning will be quite busy
for DUNE, which is currently in the stage of broad and active R\&D effort, supported by the development of appropriate 
software tools in each area. Extensive simulations are being performed to support design decisions pertaining to the Liquid 
Argon TPC as well as a number of other detector elements and subsystems (cf. the proposed Near Detector, the Photon Detector, 
the target complex etc). Background characterization, event reconstruction techniques and studies are systematic errors are all under active development.

Importantly, at the time of writing DUNE is actively working on two LArTPC prototypes:
\begin{itemize}
\item 35t Liquid Argon TPC situated at FNAL which will start taking cosmic ray data in 2016
\item The ``Full Scale'' prototype to be placed in a test beam at CERN, scheduled for commissioning in 2017 and actual data taking in early 2018
\end{itemize}

It follows that although commissioning and operation of the eventual complete detector is still years away, DUNE is effectviely engaged
in a vigorous experimental program which poses non-trivial requirements from the computing standpoint. The landscape of DUNE
computing will be defined by these and similar tasks for the next few years.

After the crucial prototyping stage is completed, the workup to start of data taking in DUNE will include scaling up the data management and processing systems,
integration testing,  stress testing and  realistic data challenges preceding the commencement of operations. For this reasons it is obvious that the DUNE
Computing Model will undergo considerable evolution. In the present document, this is addressed  by using the following approximation and structure:

\begin{itemize}

\item The time period between the year 2014 and the commissioning of DUNE (its timing being uncertain as of now but assumed to be around 2026) is divided into two parts of roughly equal duration, i.e. 2014-2020 and 2020-2026. This segments will be referred to as \textbf{T1} and \textbf{T2}.


\item Where warranted, major sections of the Computing Model will contain separate subsections reflecting these two time periods.

\end{itemize}
 
It is obvious that rapid evolution of software and computing technologies, combined with dynamic nature of R\&D in DUNE (and other  projects with potential impact on DUNE) and a variety of other factors makes precise formulation of the Computing Model and subsequent plans very difficult over the foreseen extended timeline. It  therefore likely that it will become necessary to evaluate the Computing Model and make adjustments at some point in time. We estimate that the transition between T1 and T2 periods (i.e. roughly 2020) will be an appropriate point in time to revisit the Computing Mode, develop its next
iteration and adjust the plans derived from it.

