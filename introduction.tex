\section{Introduction}
\subsection{DUNE}

The Deep Underground Neutrino Experiment (DUNE) will provide a unique, world-leading program
for the exploration of key questions at the forefront of particle physics and astrophysics~\cite{sciopps,cdr_vol2}.
The two primary components of the DUNE experimental apparatus are:

\begin{enumerate}

\item A high-resolution \textbf{Near Neutrino Detector} just downstream of the neutrino source
\item A massive liquid argon time-projection chamber (LArTPC) deployed as a \textbf{Far
Detector} deep underground and 1,300 km downstream from the source of neutrinos located at FNAL

\end{enumerate}

Three options for the DUNE Near Detector are under consideration.  The first of these is a Fine-Grained tracker (FGT) with
an electromagnetic calorimeter and muon identification.  This detector has a channel count of ${\cal O}(10^{5})$.  It is
the reference design, described in the Conceptual Design Report, Vol.~4~\cite{cdr_vol4_docdb}.  The second option is a liquid argon
TPC, to match the Far Detector technology.  The third option is a gaseous-argon TPC.  This document will focus on the FGT reference
design, and will be updated as the Near Detector system design evolves.

The Far Detector will be installed at the Sanford Underground Research Facility in Lead, South Dakota~\cite{surf}.
The fiducial mass of the liquid argon will be 40\,kt. In the reference design it is a single-phase LArTPC
with signal readout done with a number of planar arrays of wire electrodes, arranged in stereo pairs.
Approximately 1.5~million readout channels will detect ionization signal resulting from interactions in the detector. 
The Far Detector will also incorporate a photon detection system which will detect scintillation
signal due to ionizing particles traveling within the liquid argon.

The second option for the Far Detector is the dual-phase LArTPC, in which ionization electrons 
drift  in Liquid Argon, reach the surface and are extracted into the gaseous portion of the detector,
due to specially designed electrostatic field configuration. Amplification in the gaseous phase
is achieved, opening possibility for an improved signal to noise ratio.

This document will focus on the reference (single-phase) design of the Far Detector LArTPC, and will be updated
as the design evolves.

More details on the physics, design and operation of DUNE detectors 
may be found in the DUNE Conceptual Design Report Vol.4 ~\cite{cdr_vol4_docdb}. 


\subsection{About this Document}
\label{sec:modelrole}

This document contains a description of the \textit{Computing Model} (Sec.\ref{sec:computing_model}),
as well as supporting materials that help estimate the parameters of the model and
explain the policies and technical solutions that have been chosen.  These supporting materials are
organized in the following sections:

\begin{itemize}
\item DUNE Data Characteristics (Section~\ref{sec:data-characteristics}).
\item Projected scale and capability of the Far Detector DAQ (Section~\ref{sec:daq}).
\item Near Detector DAQ (Section~\ref{sec:daq-nd})
\item Software and Computing Requirements, Appendix~\ref{sec:requirements}.
\end{itemize}

\noindent
The \textit{Data Characteristics} part describes the characteristics of the data to be collected, processed and otherwise generated in the experiment.
For example, raw data rates generated by DUNE detector systems will determine the scale and required capabilities of the Data
Acquisition System (DAQ).  A brief summary of the Far Detector and Near Detector Data Acquisition is presented in Sections~\ref{sec:daq}
and~\ref{sec:daq-nd} respectively.

The characteristics of the data coming out of DAQ will determine the scale of data storage, network
and other characteristics of the distributed computing system required by DUNE.  Additionally, the scope and projected
volume of Monte Carlo studies  will affect the scale of CPU and storage resources needed by DUNE.  The scope and complexity of the DUNE
reconstruction software impacts both the resources required to process the data and the Monte Carlo, and also affect the
number and size of Monte Carlo studies required, and thus have an impact on the Computing Model.

There is a glossary of terms and abbreviations provided in Appendix~\ref{sec:appendix-glossary}. In addition,
there are ``Definitions'' subsections in the Requirements section of the document~(Sec.\ref{sec:requirements}) where
many relevant terms are defined.

The \textit{Requirements} which are included as Appendix~\ref{sec:requirements}
define a set of rules and principles by which technologies are selected and specific practices and policies implemented in DUNE computing.
The benefit of having the \textit{Requirements} in place is that they reflect a broad consensus within the Collaboration regarding what it needs from
its computing sector and its components, thus fostering cooperation and better integrations of systems and components in DUNE.

\subsection{Timeline and Evolution of the Computing Model}
According to current plans, DUNE will be commissioned in mid-2020s. The decade before the commissioning will be quite busy
for DUNE, which is currently in the stage of broad and active R\&D effort, supported by the development of appropriate 
software tools in each area. Extensive simulations are being performed to support design decisions pertaining to the Liquid 
Argon TPC as well as a number of other detector elements and subsystems (cf. the proposed Near Detector, the Photon Detector, 
the target complex etc). Background characterization, event reconstruction techniques, and studies of systematic 
uncertainties are all under active development.

At the time of writing a high priority in DUNE is assigned to two LArTPC prototypes:
\begin{itemize}
\item The ``35t prototype'' --- a Liquid Argon TPC situated at FNAL which will start taking cosmic ray data in 2016
\item The ``Full Scale'' prototype (``protoDUNE/NP04'') to be placed in a test beam at CERN, 
scheduled for commissioning in 2017 and actual data taking in early 2018
\end{itemize}
\noindent
Although commissioning and operation of the complete experiment is still years away, DUNE is
already engaged in a substantial program with significant computing needs. The
landscape of DUNE computing will be defined by these and similar tasks for the next few years. Computing plans for this prototyping
work are presented in Appendix~\ref{sec:dune-prototypes}.

While the prototyping phase is underway, the DUNE Collaboration will start preparations for data taking with the
full apparatus, including the Far Detector at the Sanford facility and the Near Neutrino Detector and its associated systems
located at FNAL.  These preparations will involve scaling up the data management and processing systems,
integration testing,  stress testing and realistic data challenges.
The DUNE Computing Model will therefore undergo considerable evolution in the following decade. In the present document,
this is addressed  by describing the computing models for the protoypes and for the final DUNE experiment in separate sections.

This Computing Model is a living document.
The rapid evolution of software and computing technologies, combined with the dynamic nature of R\&D in DUNE
and a variety of other factors will require re-evaluation and adjustment of the Computing Model. 
We anticipate the next update to be produced no later than 2020.
